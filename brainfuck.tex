\catcode`@=12

\bgroup
\def\:{\global\let\BlankSpace= }\:
\egroup

% not ignoring whitespace
\long\def\IfNextChar#1#2#3{%
\let\INCToken= #1% <- MEMO: this space is crucial
\def\INCTrue{#2}\def\INCFalse{#3}%
\futurelet\INCTok\INC%
}

\def\INC%
  {\ifx\INCTok\INCToken\let\INCFalse\INCTrue\fi\INCFalse}




% \expandafter\mathchardef\csname BFtest\endcsname=10234
% \showthe\BFtest

\expandafter\mathchardef\csname BFpointer\endcsname=0

\count255 = 0
\loop
\expandafter\chardef\csname BFreg\the\count255\endcsname=0
\ifnum\count255 < 29999
\advance\count255 by 1
\repeat

% \expandafter\showthe\csname BFreg29999\endcsname



\def\BFIncrementPointer{
\showthe\BFpointer
\edef\Act{\noexpand\mathchardef\noexpand\BFpointer=\numexpr\the\BFpointer+1\relax}
\Act
\showthe\BFpointer
}

\def\AdvancePointerBy#1{
\def\Reg{\mathchardef\BFpointer}
\edef\Val{\numexpr\the\BFpointer+#1\relax}
\Reg=\Val
}

\def\AdvanceRegisterBy#1{
\def\Reg{\expandafter\chardef\csname BFreg\the\BFpointer\endcsname}
\edef\Val{\numexpr\the\csname BFreg\the\BFpointer\endcsname+#1\relax}
\Reg=\Val
}

\def\IncrementRegister{\AdvanceRegisterBy{+1}}
\def\DecrementRegister{\AdvanceRegisterBy{-1}}

\def\IncrementPointer{\AdvancePointerBy{+1}}
\def\DecrementPointer{\AdvancePointerBy{-1}}


\def\DumpMem{
\count255 = 0
\loop
\expandafter\the\csname BFreg\the\count255\endcsname,
\ifnum\count255 < 10
\advance\count255 by 1
\repeat
}

\tt

% \def\Loop#1\Pool

\begingroup

\DumpMem

\catcode`>=\active\let>\IncrementPointer
\catcode`<=\active\let<\DecrementPointer
\catcode`+=\active\let+\IncrementRegister
\catcode`-=\active\let-\DecrementRegister
>+>>+++->+
\DumpMem

\endgroup



\begingroup
\loop
% \catcode`=\active
\ifnum\count255 < 10
\advance\count255 by 1
\repeat


\endgroup


\showthe\catcode`\

\def\gob














Hello world.

\bye

una buona idea sarebbe usare
[ = {
] = }
ed adottare un modello parser-like che rilevi gruppi








